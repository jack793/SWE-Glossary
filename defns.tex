\newglossaryentry{Engineering}
{
  name=Engineering,
  description={Applicazione di principi scientifici e matematici per fini pratici. (Non inventa niente, ma applica)}
}

\newglossaryentry{Progetto}
{
  name=Progetto,
  description={Incarico contrattuale fra parti e non più negoziabile}
}

\newglossaryentry{Business management}
{
  name=Business management,
  description={Chi fissa gli obiettivi in termini di costi, profitto, priorità strategiche}
}

\newglossaryentry{Project managment}
{
  name=Project managment,
  description={Chi gestisce le risorse di progetto e riferisce all'organizzazione e al cliente}
}

\newglossaryentry{Development team}
{
  name=Development team,
  description={Chi realizza il prodotto: il luogo di appartenenza dei software engineer}
}

\newglossaryentry{Customers}
{
  name=Customers,
  description={Chi compra il prodotto SW}
}

\newglossaryentry{End users}
{
  name=End users,
  description={Chi usa il prodotto SW}
}

\newglossaryentry{Processi di ciclo di vita}
{
  name=Processi di ciclo di vita,
  description={Aggregati ordinati di attività che vanno svolte per causare transizioni di stato nel ciclo di vita di un prodotto SW}
}

\newglossaryentry{Modelli di ciclo di vita}
{
  name=Modelli di ciclo di vita,
  description={Descrivono come i processi si relazionano tra loro nel tempo rispetto agli stati di ciclo di vita}
}

\newglossaryentry{Configurazione}
{
  name=Configurazione,
  description={Di quali parti è fatto il prodotto e in quale ordine (l'idea di un makefile descrive molto bene il termine)}
}

\newglossaryentry{Code-'n-Fix}
{
  name=Code-'n-Fix,
  description={Attività eseguite senza organizzazione preordinata}
}

\newglossaryentry{designpattern}
{
  name=Design pattern,
  description={Soluzione progettuale generale a un problema ricorrente. Una descrizione o un modello da applicare per risolvere un problema che può presentarsi in diverse situazioni durante la progettazione e lo sviluppo del software}
 }
 
\newglossaryentry{documentazionediprogetto}
{
  name=Documentazione di progetto,
  description={Tutto ciò che descrive gli ingressi e le uscite delle attività necessarie al progetto}
}

\newglossaryentry{repository}
{
  name=Repository,
  description={Database centralizzato nel quale risiedono, individualmente, tutti i configuration item di ogni \textit{baseline} nella loro storia completa}
}

\newglossaryentry{attributidiprodotto}
{
  name=Attributi di prodotto,
  description={All'interno della classificazione dei requisiti, definiscono le caratteristiche richieste dal sistema.\\ Rispondono alla domanda: Cosa devo fare? (requisiti funzionali, prestazioni, di qualità del prodotto)}
}

\newglossaryentry{attributidiprocesso}
{
  name=Attributi di processo,
  description={All'interno della classificazione dei requisiti, pongono vincoli sui processi impiegati nel progetto 
  ..\\ Rispondono alla domanda: Come devo farlo? (requisiti di vincoli realizzativi, normativi e contrattuali)}
}

\newglossaryentry{requisito}
{
  name=Requistito,
  description={Definizioni multiple:
  \begin{enumerate}
  \item Condizione necessaria ad un utente per risolvere un problema o raggiungere un obiettivo (visione dal lato del bisogno)
  \item Condizione che deve essere soddisfatta da un sistema per adempiere ad un obbligo (visione dal lato della soluzione)
  \end{enumerate}}
}

\newglossaryentry{Verifica}
{
  name=Verifica,
  description={Accertare che l'esecuzione delle attività di processo non abbia introdotto errori (Ho costruito il sistema correttamente?)}
}

\newglossaryentry{validazione}
{
  name=Validazione,
  description={Accertare che il prodotto realizzato corrisponde alle attese (\' E un sistema corretto?)}
}

%%%%%%%%%%%%%%%%%%%%%%% ** DA CONTROLLARE ** %%%%%%%%%%%%%%%%%%%%

\newglossaryentry{Sistematico}
{
name=Sistematico,
description={Colui che sa far quella cosa - darsi delle regole - approcciarsi in al problema con metodo. Ciò contribuisce all'efficienza e all'efficacia}
}

\newglossaryentry{Stakeholder}
{
name=Stakeholder,
description={Dall'inglese, portatore d' interesse. È l'insieme di persone a vario titolo coinvolte nel ciclo di vita del Software con influenza sul prodotto}
}

\newglossaryentry{Processo}
{
name=Processo, 
description={Multiple definizioni:
\begin{enumerate}
\item Detto in inglese come \textit{Way of working}. Esso è rappresentabile come un automa a stati, dove ogni stato rappresenta uno stadio del ciclo di vita del processo
\item Un insieme di attività interconnesse, che trasforma uno o più input in output consumando risorse. [SWEBok 8-1]
\end{enumerate}
}
}

\newglossaryentry{Procedura}
{
name=Procedura,
description={Una procedura è un ordinato insieme di passi o, alternativamente, controlli del lavoro per eseguire il task}
}


\newglossaryentry{Sink}
{
name=Sink,
description={Dall'inglese "Punto di fine", scarico da cui non esco più (ovvero uno stato finale)}
}

\newglossaryentry{Source}
{
name=Source,
description={Dall'inglese, stato che ammette solo archi in uscita}
}

\newglossaryentry{ciclovitasoftware}
{
name=Ciclo di vita del software,
description={Sono gli stati che il prodotto assume dal concepimento al ritiro}
}

\newglossaryentry{efficacia}
{
name=Efficacia,
description={Definizioni multiple:
\begin{enumerate}
\item Conformità al contratto. Si garantisce ciò che si deve fare, ed è determinata dal grado di conformità del progetto rispetto alle norme vigenti e agli obiettivi prefissati. L'efficacia è direttamente proporzionale alla quantità di risorse impiegate.
\item L'efficacia è il rapporto tra l'output attuale e quello attesso, prodotto dal processo, attività o compito [SWEBok-v3 8.4.1].
\end{enumerate}
}
}

\newglossaryentry{efficienza}
{
name=Efficienza,
description={Definizioni multiple: 
\begin{enumerate}
\item L'efficienza è la capacità di azione o di produzione con il minimo di scarto, di spesa, di risorse e di tempo impiegati. \`E inversamente proporzionale alla quantità di risorse impiegate nell'esecuzione delle attività richieste.
\item L'efficienza è il rapporto tra le risorse consumate e quelle attese o desiderate nel compiere un processo, attività o compito [SWEBok-v3 8.4.1].
\end{enumerate}
}
}

\newglossaryentry{bestpractice}
{
name=Best practice,
description={Prassi (modo di fare) che per esperienza e per studio abbia mostrato di garantire i migliori risultati in circostanze note e specifiche}
}

\newglossaryentry{iterazione}
{
name=Iterazione,
description={Procedere per iterazioni significa operare raffinamenti o rivisitazioni. Essa è associabile a un'operazione, ad un qualcosa fatto prima (già fatto). Questa operazione è potenzialmente distruttiva e ha caratteristiche molto pericolose, perchè non sa garantire come finirà ed è una ripetizione di una cosa che ho già fatto},
plural=iterazioni
}

\newglossaryentry{incremento}
{
name=Incremento,
description={Avvicinamento alla meta che si compie in due modi: aggiungendo o togliendo. Procedere per incrementi significa aggiungere a un impianto base. Un incremento non può mai tornare sui suoi passi, ed è preferibile rispetto alla iterazione, perchè pianifica i passi e ciò significa che si arriverà a una fine},
plural=incrementi
}

\newglossaryentry{prototipo}
{
name=Prototipo,
description={Originario, abbozza, serve per capire se si sta andando in una direzione giusta o no. Esistono due tipi di prototipi: usa e getta da usare solamente se il beneficio è molto maggiore del costo per produrla, altrimenti se si presta ad essere la soluzione, anche se può essere una base per una iterazione},
plural=prototipi
}
\newglossaryentry{riuso}
{
name=Riuso,
description={Due tipi di riuso: opportunistico (in stile copy-paste, a basso coso ma scarso impatto), altrimenti l'altro uso è quando si sa cosa prendo, so perchè lo prendo e so cosa fa. Fare software è fondalmentalmente riuso. \`E quindi una delle attività più importanti di SWE, assume una connotazione positiva},
plural=riusi
}

\newglossaryentry{controllodiversione}
{
name=Controllo di versione,
description={--Da completare--}
}

\newglossaryentry{disciplinato}
{
name=Disciplinato,
description={Saper prevedere i costi. Avere una quantità credibile, seguendo le regole. Essere disciplinati significa anche seguire un ordine preciso degli stati nel ciclo di vita del software},
plural=disciplinati
}

\newglossaryentry{controllodeiprocessi}
{
name=Controllo dei processi,
description={Luogo in cui si pongono delle regole per essere sempre efficaci e disciplinati}
}

\newglossaryentry{trigger}
{
name=Trigger,
description={Evento che causa il cambiamento di arco nel ciclo di sviluppo del software. Attività la quale fa cambiare lo stato dell'automa}
}

\newglossaryentry{fase}
{
name=Fase,
description={Durata temporale entro uno stato di ciclo di vita o in una transizione tra essi}
}


\newglossaryentry{pre-condizione-cascata}
{
name=Pre-condizione,
description={Nel modello a cascata, la pre-condizione è ciò che è verificato prima di entrare in un certo stato}
}

\newglossaryentry{post-condizione-cascata}
{
name=Post-condizione,
description={Nel modello a cascata, è ciò che dev'essere vero dopo lo svolgimento delle attività}
}

\newglossaryentry{meta-modello}
{
name=Meta-modello,
description={Insieme di regole, vincoli e teorie utilizzate per la modellazione di una classe di problemi con astrazione dal mondo reale}
}

\newglossaryentry{casoduso}
{
name=Caso d'uso,
description={Tecniche per individuare i requisiti funzionali. Queste Tecniche devono essere comprensibili anche all'utente committente. Il caso d'uso descrive l'insieme di funzionalità del sistema come sono percepite dagli utenti}
}

\newglossaryentry{scenario}
{
name=Scenario,
description={Rappresenta una sequenza di passi che descrivono iterazioni tra gli utenti e il sistema}
}

\newglossaryentry{attore}
{
name=Attore,
description={Elemento esterno al sistema che interagisce con il sistema}
}

\newglossaryentry{progetto}
{
name=Progetto,
description={Insieme di tre elementi importanti: \begin{enumerate}
\item Insieme ordinato di compiti da svolgere
\item I compiti da svolgere sono pianificati da inizio a fine
\item I vincoli che vengono tenuti conto quando si pianifica nascono da quanto tempo ho a disposizione per l'intero progetto e quali strumenti \`e possibile utilizzare per dare i risultati attesi
\end{enumerate}
}
}

\newglossaryentry{rischio}
{
name=Rischio,
description={\`E il non aver tenuto conto che le cose possono non andare come avevamo considerato}
}

\newglossaryentry{slack}
{
name=Slack,
description={Margine tra inizio e fine di un'attività. In italiano è sinonimo di lasco}
}

\newglossaryentry{primaryprocess}
{
name=Processo primario,
description={I processi primari includono processi sofware per: \begin{itemize}
\item Sviluppo
\item Operazioni o funzioni
\item Mantenimento del software
\end{itemize}
[def. SWEBok-v3 8-2.1.3]}
}

\newglossaryentry{supportinprocess}
{
name=Processo di supporto,
description={Processi di supporto sono applicati discontinuamente o continuamente durante il ciclo di vita del software a supporto dei processi primari; questi includono:\begin{itemize}
\item Gestione configurazione
\item Controllo della qualit\`a
\item Verifica e validazione
\end{itemize}
[def. SWEBok-v3 8-2.1.3]}
}

\newglossaryentry{organizationalprocess}
{
name=Processo organizzativo,
description={I processi organizzativi provvedono al supporto all'ingegneria del software. Includono: \begin{itemize}
\item Formazione
\item Analisi di misura del processo
\item Gestione dell'infrastruttura
\item Portfolio e riuso
\item Organizzazione miglioramento dei processi
\item Gestione del modello del ciclo di vita del software
\end{itemize}
[def. SWEBok-v3 8-2.1.3]}
}

\newglossaryentry{SDLC}
{
name=Ciclo di vita dello sviluppo software (SDLC),
description={Un ciclo di vita dello sviluppo software include i processi software usati per specificare e trasformare requisiti software in un prodotto software finito [def. SWEBok-v3 8-2]}
}

\newglossaryentry{SPLC}
{
name=Ciclo di vita del prodotto software (SPLC),
description={Un ciclo di vita del prodotto software include un SDLC più addizionali processi software che provvedono al: \begin{itemize}
\item distribuzione
\item mantenimento
\item supporto
\item evoluzione
\item ritiro
\end{itemize}
e tutti gli altri processi di inizio al ritiro, includendo processi di gestione per il controllo della configurazione e della qualità applicati durante il ciclo di vita del prodotto software [def. SWEBok-v3 8-2]}
}

\newglossaryentry{camminocritico}
{
name=Cammino critico,
description={Sequenza di attività-progetto che ha lo slack più piccolo}
}

\newglossaryentry{configuration}
{
name=Configuration,
description={Definizioni multiple:
\begin{enumerate}
\item La configuration si basa sul concetto di sistema e la si ha dall'inizio di uno sviluppo (conception) fino alla fine (uso operativo). Ogni pezzo del sistema ha il suo perchè e della conception fino all'uso operativo ha diverse configuration. Si hanno tante configuration  potenzialmente in base alle configurazioni che si avranno, e si deve decidere quando averranno i cambiamento di scarsa o molta importanza. Questa decisione avviene attraverso le milestone.
\item Una "Software Configuration" è l'insieme delle funzionalità e delle caratteristiche di hardware o software così come indicate nella documentazione o raggiunte in un prodotto. [SWEBok 6-6]
\end{enumerate}
}
}

\newglossaryentry{configurationitem}{
name=Configuration item,
description={Definizioni multiple:
\begin{enumerate}
\item Un configuration item è un elemento o un'aggragazione di hardware e/o software che può essere gestito come una singola entità. [SWEBok 6-6]
\item Un configuration item è qualsiasi cosa associata ad un progetto software (progettazione, codice, dati di test, documentazione) che sia stato messo sotto un controllo di configurazione. Spesso un configuration item ha diverse versioni, e ha un nome univoco. [Sommerville, pag 684]
\end{enumerate}
}
}

\newglossaryentry{softwareconfigurationitem}{
name=Software configuration item,
description={Un software configuration item è un'entità software che è stata stabilita come configuration item.[SWEBok 6-6]}
}

\newglossaryentry{SCM}{
name=Software Configuration Management (SCM),
description={L'SCM è un processo di supporto al ciclo di vita di un software che avvantaggia la gestione di progetto, le attività di sviluppo e manutenzione, l'attività di garanzia di qualità, così come gli utenti e i clienti del prodotto finale.[SWEBok 6-1]}
}

\newglossaryentry{milestone}
{
name=Milestone,
description={Le milestone servono per fissare dei punti di avanzamento significativi rispetto agli obiettivi stabiliti e al tempo a disposizione.
Un progettatore assegna milestone che hanno una distanza tale per cui arrivarci significa raggiungere un punto importante: infatti, ogni milestone corrispone a una specifica configurazione del sistema.
Ogni milestone ha un proprio nome se associata e una configurazione detta \textit{baseline}}
}

\newglossaryentry{baseline}
{
name=Baseline,
description={Definizioni multiple:
\begin{enumerate}
\item La baseline è un punto di avanzamento certo, dal quale non si torna mai indietro. Viene visto come un punto di situazione certa dalla quale si potrà soltanto avanzare senza mai retrocedere.
\item La baseline è una versione approvata di un configuration item che è stata formalmente progettata e definita (/sistemata, "fixed") in un momento specifico del ciclo di vita del configuration item. [SWEBok 6-7]
\item Una baseline è una collezione delle versioni dei componenti che compongono un sistema. Le baseline sono controllate, il che significa che le versioni dei componenti che compongono il sistema non possono essere cambiate e che è sempre possibile ricreare una baseline a partire dai componenti che la costituiscono. [Sommerville, pag 684]
\end{enumerate} }
}
\newglossaryentry{Checkin}
{
  name=Check-in,
  description={All'interno di una \textit{repository} premette a ciascun membro del team di lavorare su vecchi e nuovi CI senza rischio di sovrascritture accidentali}
}

\newglossaryentry{Checkout}
{
  name=Check-out,
  description={Attività che permette a ciascun membro del team di condividere il lavorato nello spazio comune della \textit{repository}}
}

\newglossaryentry{versione}{
name=Versione,
description={Definizioni multiple:
\begin{enumerate}
\item La versione di un elemento software è un'istanza identificata dell'elemento stesso. Può essere pensata come uno stato di un elemento in evoluzione.[SWEBok 6-7]
\item Una versione è un'istanza di un configuration item che differisce, in qualche modo, dalle altre istanze di quell'item. Le versioni hanno sempre un'identificatore unico, che spesso è composto dal nome del configuration item più un numero di versione. [Sommerville, pag 684]
\end{enumerate}
}
}

\newglossaryentry{controllodiconfigurazione}{
name=Controllo di configurazione,
description={\`E il processo che garantisce che le versioni di un sistema e i componenti siano registrati e mantenuti in modo da poter gestire i cambiamenti e poter identificare e memorizzare tutte le versioni dei componenti durante il tempo di vita del sistema. [Sommerville, pag 684]}
}

\newglossaryentry{codeline}{
name=Codeline,
description={\`E un insieme di versioni di un componente software e dei configuration item dai quali dipende. In altre parole è una sequenza di versioni di codice sorgente nella quale le versioni successive derivano dalle precedenti. [Sommerville, pag 684/690]}
}

\newglossaryentry{mainline}{
name=Mainline,
description={\`E una sequenza di baseline che rappresenta le differenti versioni di un sistema. [Sommerville, pag 684]}
}

\newglossaryentry{release}{
name=Release,
description={\`E una versione di un sistema rilasciata ai consumatori. [Sommerville, pag 684]}
}

\newglossaryentry{workspace}{
name=Workspace,
description={\`E un'area di lavoro privata dove il software può essere modificato senza influenze da parte degli altri sviluppatori che stanno modificando lo stesso software. [Sommerville, pag 684]}
}

\newglossaryentry{branching}{
name=Branching,
description={\`E la creazione di una nuova codeline a partire da una esistente. Le due codeline possono essere sviluppate in modo indipendente. [Sommerville, pag 684]}
}

\newglossaryentry{merging}{
name=Merging,
description={\`E la creazione di una nuova versione di un componente ottenuta unendo versioni separate in codeline differenti. Queste codeline possono essere state create da una precedente ramificazione (branch).  [Sommerville, pag 684]}
}

\newglossaryentry{systembuilding}{
name=System building,
description={\`E la creazione di una nuova versione eseguibile del sistema attraverso la compilazione e il "linkaggio" di versioni appropriate dei componente e delle librerie che compongono il sistema. [Sommerville, pag 684]}
}

\newglossaryentry{stimaqualitativa}{
name = Stima qualitativa,
description = La stima si basa sul giudizio di esperti [SWEBok - 8.3.2]
}

\newglossaryentry{stimaquantitativa}{
name = Stima quantitativa,
description = La valutazione della stima la si assegna attraverso un punteggio sulla base delle analisi di risultati che indicano il raggiungimento dell'obiettivo e l'esito di un processo definito [SWEBok - 8.3.2]    
}

\newglossaryentry{classificazionefasi}{
name = Classificazione a fasi, 
description = La classifcazione di un processo software è stabilita assegnando la stessa valutazione di maturità a tutti i processi all'interno di uno specifico livello [SWEBok - 8.3.4]
}

\newglossaryentry{classificazionecontinua}{
name = Classificazione continua,
description = La classificazione avviene assegnando una valutazione ad ogni processo d'interesse [SWEBok - 8.3.4]
}

\newglossaryentry{produttivita}{
name = Produttività,
description = {Il rapporto tra l'output prodotto e le risorse consumate [SWEBok-v3 8.4.1], ovvero \[\frac{efficacia}{efficienza}\]}
}

\newglossaryentry{tracciamento}{
name = Tracciamento,
description = {Procedimento tramite il quale per ogni baseline si sa ciò che si è fatto e perchè e si conosce la qualità del lavoro svolto [def. Prof. Vardanega]}
}

\newglossaryentry{framework}{
name = Framework,
description = {(in italiano: quadro di lavoro) Insieme di regole che costruiscono una soluzione coerente [def. Prof. Vardanega]}
description = {Struttura di supporto su cui un software può essere organizzato e progettato [def. Prof. Cardin]}
}
